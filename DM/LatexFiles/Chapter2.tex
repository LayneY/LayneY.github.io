\documentclass{article}

\usepackage{amsmath, array, amssymb, mathtools} 

\title{Chapter 2 Homework}
\author{Layne Yarbrough}
\date{September 11, 2023}

\DeclarePairedDelimiter{\ceil}{\lceil}{\rceil}

\begin{document}
\maketitle
Problem 2.1
\begin{align*}
  -1 &= 2k+1\\
  -2 &= 2k\\
  -1 &= k
\end{align*}
An odd number is any number that can be represented by $2k+1$, where $k$ is any real integer. Since this is true for -1, -1 is an odd number.\\

Problem 2.3
\begin{align*}
  a &= 2n+1\\
  b &= 2m+1\\
  ab &= (2n+1)(2m+1)\\
  ab &= (4nm + 2n + 2m + 1)\\
  ab &= 2(nm + n + m) + 1\\
  k &= 2nm + n + m\\
  ab &= 2k+1
\end{align*}
As long as $m, n \in \mathbb{Z}$, and assuming $a$ and $b$ are odd numbers, than $ab$ will also be an odd number.\\

Problem 2.5
\begin{align*}
  \sqrt[3]{2} &= \frac{a}{b}\\
  b * \sqrt[3]{2} &= a\\
  2b^3 &= a^3\\
  2b^3 &= (2k)^3\\
  (2k)^3 &= 8k^3\\
  b^3 &= 4k^3
\end{align*}
Assume for contradiction that $\sqrt[3]{2}$ is rational. This means that it must be a ratio of two numbers. At least one of $a$ or $b$ must be odd, otherwise, if they were both even, the ratio $\frac{a}{b}$ could be simplified by a factor of 2. When we get to $2b^3 = a^3$, we know that $a$ must be even. When we get to the final step, we can see that $b$ must be divisible by 4, so it must also be even. This goes against our original statement that at least one of $a$ or $b$ must be odd, so $\sqrt[3]{2}$ cannot be rational, therefore it is irrational.\\

Problem 2.7\\
If a die has seven sides, the sides should have equal surface area. Therefore, the chance of the die landing on any one side is equal to each of the other sides.\\

Problem 2.9 (a)
\begin{align*}
  a, b &\in \mathbb{Z}\\
  c &= a^2\\
  d &= b^2\\
  a &= p_1^{x_1}, p_2^{x_2}, ... p_n^{x_n}\\
  b &= q_1^{x_1}, q_2^{x_2}, ... q_m^{x_m}\\
  a^2 &= p_1^{2x_1}, p_2^{2x_2}, ... p_n^{2x_n}\\
  b^2 &= q_1^{2x_1}, q_2^{2x_2}, ... q_m^{2x_m}\\
  cd &= a^2 * b^2\\
  cd &= a^2b^2\\
  a^2b^2 &= p_1^{2x_1} * q_1^{2x_1} * p_2^{2x_2} * q_2^{2x_2} * ... * p_n^{2x_2} * q_n^{2x_m}\\
  cd &= (ab)^2
\end{align*}
Since all the exponents are even, a 2 can be factored out so the result is a perfect square.\\

Problem 2.9 (b)
\begin{align*}
  c &= 12\\
  d &= 3\\
  cd &= (12)(3)\\
  cd &= 36
\end{align*}
This proves that this statement is not true, because neither 12 or 3 are perfect squares and $12 \neq 3$. However, in this example $cd = 36$ is a perfect square. Therefore the statement is false because if $cd$ is a perfect square, $c$ and $d$ are not necessarily perfect squares all the time.\\

Problem 2.9 (c)
\begin{align*}
  x, y &\in \mathbb{Z}\\
  c &> d\\
  c &= x^2\\
  d &= y^2\\
  x^2 &> y^2\\
  \sqrt{x^2} &> \sqrt{y^2}\\
  x &> y
\end{align*}
The number that is greater must also have a greater square.\\

Problem 2.11\\
The line $(x+y)(x-y) > 0 $ cannot be changed to $x+y > 0$, and the result should have been $x^2-y^2 > 0$.\\

Problem 2.13\\
(a) For any positive real number $n$, $n$ has two square roots, $x$ and $y$, such that $x \neq y$.\\
(b)For all positive numbers $n$, such that $n = 2k$, where $k$ is any real integer, $n=x+y$, such that $x$ an $y$ are both prime.\\

Problem 2.15\\
\begin{align*}
  \ceil*{\frac{5}{3}} &\approx \ceil*{1.67}\\
  \ceil*{1.67} &= 2
\end{align*}
There are 2 people left out, so X must know the other 3 or not know the other 3 out of the 5.

\end{document}
